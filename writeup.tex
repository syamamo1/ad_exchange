\documentclass[12pt,letterpaper]{article}
\usepackage{fullpage}
\usepackage[top=2cm, bottom=4.5cm, left=2.5cm, right=2.5cm]{geometry}
\usepackage{amsmath,amsthm,amsfonts,amssymb,amscd}
\usepackage{lastpage}
\usepackage{enumerate}
\usepackage{fancyhdr}
\usepackage{mathrsfs}
\usepackage{graphicx}
%% download this package and put it in the same directory as this file
%\usepackage{clrscode3e}
\setlength{\parindent}{0.0in}
\setlength{\parskip}{0.05in}

% Edit these as appropriate
\newcommand\course{CS1420}
\newcommand\semester{Spring 2023}            % <-- current semester
\newcommand\hwnumber{1}                    % <-- homework number
\newcommand\pbnumber{1}                    % <-- problem number
\newcommand\BannerID{B01326006}           % <-- Your Banner ID!

\pagestyle{fancyplain}
\headheight 35pt
\lhead{\BannerID}
\chead{\textbf{\Large Homework \hwnumber}}
\rhead{\course\;--\;\semester \\ \today}
\lfoot{}
\cfoot{}
\rfoot{\small\thepage}
\headsep 1.5em

\begin{document}

\section*{Ad Exchange Algorithm}
Our strategy is to calculate a ratio that we shade our bids by (bid = ratio * 1.0). 
We split the market segments into two different classes based on if they have
2 specifying attributes (Female HighIncome) or 3 (Female Young HighIncome).
For both the 2 and 3 attribute segments, ratio is calculated using:

$$r=\frac{9}{20}\cdot \frac{n_{sub}}{n_{pos}}$$

9 is the number of competitors so we need to bid more when there is more
competition. 20 is the number of total classes. The more classes, the less
we have to bid because there's a lesser chance the competition will bid 
on the users we bid for. $n_{sub}$ is the amount of users that belong to 
one of the 2 or 3-attribute segments that are subsets of our segment. This number
measures how many options other bidders with overlapping segments have. If
they have more options, we don't have to bid as much as there is more supply. 
$n_{pos}$ is the amount of users in the segments that our segment is a subset of. 
This is the amount of users we can possibly win and the more there is, the more
supply and the less we have to bid. 

For spending limit, we alter our strategy for the 2 attribute and 3 attribute 
segments. We will show the 3 attribute equation first as it is simpler.
For the 3 attribute segments, we calculate spending limit using:

$$L_3 = 0.9\cdot r\cdot b$$

Here $b$ is the campaign budget. We multiply by our ratio $r$ because we 
know the maximum we will spend is $r\cdot b$. We also multiply by 0.9 
because we are participating in a second price auction and we assume for each
client we win, we pay 90\% of our bid. This way, we aren't spending money on
users when we have already attained our reach. For the 2 attribute segments we use:

$$L_2 = 0.9\cdot r\cdot \frac{n_{here}}{n_{pos}}\cdot b$$

The only thing we changed here is we set the limit for each segment equal to
the fraction of users in the current segment and the total possible users
we can bid on: $\frac{n_{here}}{n_{pos}}$. We don't consider the delta coefficient because it doesn't affect our demand for the user.
Our spending limit is already set up such that we don't buy too many users so our bid is 
solely determined by the level of competition and the supply for the user. 

\end{document}